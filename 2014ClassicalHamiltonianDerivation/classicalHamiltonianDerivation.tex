%\documentclass[aps,prd,twocolumn,floatfix]{revtex4}   % style for Physical Review B and AJP are similar
\documentclass[twocolumn,floatfix,nofootinbib]{revtex4}   % style for Physical Review B and AJP are similar
%\documentclass[12pt,aps,prb,preprint]{revtex4}

\usepackage{bbm}
\usepackage{amsmath}
\usepackage{amssymb}
\usepackage{graphicx}
\usepackage{amsfonts}
\usepackage{amsthm}

\theoremstyle{theorem}
\newtheorem{thm}{Theorem}[section]
\newtheorem{cor}[thm]{Corollary}
\newtheorem{lem}[thm]{Lemma}
\newtheorem{sublem}[thm]{Lemma}
\newtheorem{prop}[thm]{Proposition}
\newtheorem{ax}[thm]{Axiom}
\newtheorem{conj}[thm]{Conjecture}
\newtheorem{clm}[thm]{Claim}
\newtheorem{lemdef}[thm]{Lemma-Definition}


\theoremstyle{definition}
\newtheorem{defn}[thm]{Definition}
\newtheorem{rem}[thm]{Remark}
\newtheorem{ques}[thm]{Question}
\newtheorem{cons}[thm]{\rm\bfseries{Construction}}
\newtheorem{exm}[thm]{Example}
\newtheorem{conds}[thm]{Condition}
\newtheorem{proper}[thm]{Property}
\newtheorem{defnlem}[thm]{Definition-Lemma}
\newtheorem{assump}[thm]{Assumption}
\newtheorem{warn}[thm]{Warning}
\newtheorem{situ}[thm]{Situation}

\begin{document}

\title{Deriving Hamiltonian mechanics from physics, without passing through Lagrangians.}
\author{Gabriele Carcassi}
\affiliation{Brookhaven National Laboratory, Upton, NY 11973}
\email{carcassi@bnl.gov}
\date{August 25, 2013}

\begin{abstract}
We derive the Hamiltonian formulation of classical mechanics directly, without reference to Lagrangian mechanics.
We start from the definition of states and the labels used to identify them, and show how simple properties required
on those labels lead to the Hamiltonian formulation.\end{abstract}

\maketitle

\section{Introduction}
In physics Hamiltonian mechanics is usually presented as a reformulation of Lagrangian mechanics, which is how it was originally developed. In this work we aim to derive it on its own, starting from simple physical assumptions, without any reference to Lagrangian mechanics. We believe that provides more direct insight in the theory and a direct physical interpretation of its geometric features.



Hamiltonian is either math or physics from Lagrangian. Lagrangian are not well motivated anyway.

We start from scratch and find Hamiltonian. $dt \wedge dE$

\section{States}
\textbf{State definition}. Let's assume  we have a physical system that can be found in different configurations. Each unique configuration, or configuration state, is identified by a label. For our purposes, it does not matter what it is (a measurement, a way to prepare the system, future behavior, ...), just that it exists.

\begin{defn}\label{statedef}
Let $I$ be a set of labels. Let $\mathbbm{C}_{I}$ be the set of all possible configuration states of our system identified by the labels. $\forall i \in I, \exists \mathbbm{c}_{i} \in \mathbbm{C}_{I}$.
\end{defn}

The following definitions allow us to use multiple labels.

\begin{defn}\label{combine label}
Let $I_1$ and $I_2$ be two sets of labels. We call $I = \langle I_1, I_2 \rangle$ a combined set of labels if an injective map exists such that $\forall i \in I \rightarrowtail i_1 \in I_1, i_2 \in I_2$.
\end{defn}

\begin{defn}\label{orth}
The two label sets $I_1$ and $I_2$ are said to be orthogonal iff $I = \langle I_1, I_2 \rangle = I_1 \times I_2$.
\end{defn}

The following assumptions and definitions allow us to describe a configuration state by its parts.

\begin{assump}\label{classical}
(\emph{Classical assumption}) The system is infinitesimally reducible: it can be thought as composed by two or more similar but smaller systems, each in its own configuration state, which also can be thought as composed by two or more, ad infinitum.
\end{assump}

\begin{defn}\label{statedef}
Let $\mathbbm{S}_I$ be the set of all possible configuration states of the infinitesimal subdivision. We call this set phase space. We call each element $\mathbbm{s}_i$ a state.
\end{defn}

\begin{cor}\label{statedistr}
Each configuration state $\mathbbm{c}_j \in \mathbbm{C}_J$ can be identified by a distribution over states: $\mathbbm{c}_j=\sum D(i) \mathbbm{s}_i$, where $D:I\rightarrow\mathbb{R}$ measures how much of the system can be found in each $\mathbbm{s}_i$. In physics term, the distribution can be visualized as a histogram over the discrete labels.
\end{cor}

Under the classical assumption, we can then limit ourselves to study the labels of states and their properties without losing generality.

\textbf{State mapping}. We will now concentrate ourselves on deterministic and reversible evolution.
\begin{assump}\label{detAss}
The system undergoes deterministic (future state identified by the present state) and reversible (past state identified by the present state) evolution.
\end{assump}

\begin{prop}\label{detMap}
Let $I$ be a set of labels of a system that undergoes deterministic and reversible evolution. There exists a bijective map $f:I \leftrightarrow I$.
\end{prop}

\begin{cor}\label{discreteEv}
The evolution of a reducible configuration state $\mathbbm{c}_j=\sum D(i) \mathbbm{s}_i$ under a bijective map is given by $\mathbbm{c}_{j'}=\sum D'(i) \mathbbm{s}_{i}=\sum D(f^{-1}(i)) \mathbbm{s}_{i}$.
\end{cor}

Mathematically, studying \ref{detAss} means studying bijective maps. The evolution of a distribution simply moves the elements around: the bars of the discrete histogram move place, but keep the same height.

\begin{defn}\label{labelsIndep}
Let $I_1$ and $I_2$ be two orthogonal sets of labels and $I = I_1 \times I_2$. Let $f:I \leftrightarrow I$ be a bijective map. The labels are said independent iff we can define two sets of orthogonal labels $I_1'$ and $I_2'$ such that we can define separate mappings:
\begin{enumerate}
\item $I_1' \times I_2' = I$
\item there exists a bijective map $f_1: I_1 \leftrightarrow I_1'$
\item there exists a bijective map $f_2: I_2 \leftrightarrow I_2'$
\item $f(i) = \langle f_1(i_1), f_2(i_2)\rangle$
\end{enumerate}
\end{defn}

\textbf{State counting}. We now look at how the cardinality of states evolves under a bijective map.

\begin{cor}\label{labelsCount}
Let $\hat{I} \subset I$ of finite size $n(\hat{I})$. Let $f(\hat{I})$ and $f^{-1}(\hat{I})$ be respectively the forward and backward image of a bijective map. We then have $n(\hat{I})=n(f(\hat{I}))=n(f^{-1}(\hat{I}))$, the number of states is conserved.
\end{cor}

\begin{cor}\label{labelsMultiCount}
Let $\hat{I_1} \subset I_1$ and $\hat{I_2} \subset I_2$ of finite size $n(\hat{I_1})$ and $n(\hat{I_2})$. Let $f(\hat{I_1})$, $f(\hat{I_2})$, $f^{-1}(\hat{I_1})$ and $f^{-1}(\hat{I_2})$ be the corresponding forward and backward images. We then have $n(\hat{I_1}\times\hat{I_2})=n(\hat{I_1})n(\hat{I_2})=n(f(\hat{I_1})) n(f(\hat{I_2}))=n(f^{-1}(\hat{I_1})) n(f^{-1}(\hat{I_2}))=n(\hat{f(I_1)}\times\hat{f(I_2)})=n(f^{-1}(\hat{I_1})\times f^{-1}(\hat{I_2}))$, the total number of states is conserved and remains the product of the number of labels in each set.
\end{cor}

These statements, properly generalized for the continuous case, will allow us to define a metric. Conserving that metric will lead to Hamiltonian mechanics.

\section{Numeric labels}

\textbf{Discrete labels over $\mathbb{R}$}. Labels often will be numbers. For integers, we simply use $\mathbb{Z}$ as our set of labels. For real number, we need to update our definitions.

\begin{defn}\label{disclabelsoverr}
Consider a continuous numeric range. We divide the full range into contiguous cells. Let $I$ be the set of cells. For each cell we have a center value $x: I \mapsto \mathbb{R}$ and a width $w: I \mapsto \mathbb{R}$. $I$ is a set of discrete labels over a continuous range.
\end{defn}

\begin{prop}\label{disclabelsoverrdist}
Each configuration state $\mathbbm{c}_j \in \mathbbm{C}_J$ can be identified by a distribution over states: $\mathbbm{c}_j=\sum D(i) \mathbbm{s}_i=\sum \rho(i) w(i) \mathbbm{s}_i$, where $D:I\rightarrow\mathbb{R}$ is defined as before, and $\rho(i)\equiv D(i) / w(i)$ is the density of the distribution for the cell. In physics term, the distribution can be visualized as a histogram where $w(i)$, $\rho(i)$ and $D(i)$ are respectively the width, height and area of the each bin.
\end{prop}

\begin{cor}\label{discreteEv}
Let $f: I \leftrightarrow I$ be a bijective map, we have $\mathbbm{c}_j=\sum D'(i) \mathbbm{s}_i=\sum \rho'(i) w(i) \mathbbm{s}_i = \sum D(f^{-1}(i)) \mathbbm{s}_i = \sum \rho(f^{-1}(i)) w(f^{-1}(i)) \mathbbm{s}_i$. $\rho'(i) = \rho(f^{-1}(i)) w(f^{-1}(i)) / w(i)$.
\end{cor}

The area moves from one cell of the histogram to the other. The height need to be adjusted if the cell is of a different size.

\begin{defn}\label{discreteHomogeneous}
A set $I$ of discrete labels over a continuous range is said to be homogeneous if $w(i)=k$: the bins are of equal width.
\end{defn}

With homogeneous labels no adjustment is needed, and the range can be used as a measure of the number of labels.

\textbf{Continuous labels over $\mathbb{R}$}. We now make the bin width arbitrarily small.

\begin{defn}\label{continuousLabels}
A state variable $X$ is the continuous limit of a set $I$ of discrete labels over a continuous range.
\end{defn}

To prepare for the limit we define $m(i)=w(i)/\Delta w$, where $\Delta w$ represents the average width of the cells. We increase the number of the cells and reduce $\Delta w$ such that $max(m(i))$ is bounded. In the limit we'll have a cell for each value, so we can use $x(i)$ (or simply $x$) instead of $i$ for the label. $\rho$ and $m$ will converge to functions defined over $x$. The corresponding configuration state will become $\mathbbm{c}_j=\int \rho(x) m(x) \mathbbm{s}_x dx$.

\begin{prop}\label{continuousMapping}
Let $f: X \leftrightarrow X$ be a bijective map on a state variable. The mapping must be continuous.
\end{prop}
%http://en.wikipedia.org/wiki/Limiting_density_of_discrete_points
Assume mapping is discontinuous at point $x$. Consider the cell at $x$ of width $m(x)dx$. The cell would be split into two, so it would not be mapped to one and only one cell.

\begin{prop}\label{widthMapping}
Let $f: X \leftrightarrow X$ be a bijective map on a state variable, and $x'=f(x)$. Then $\frac{dx'}{dx} = \frac{m(x')}{m(x)} $. If $X$ is homogeneous, then $dx' = dx$ and the range gives us a measure of the cardinality of labels.
\end{prop}
The mapping must be done so that the width of the cells is mapped as well, not just the center value. The width of the transported cell $m(x)dx \rightarrow m(x) \frac{dx'}{dx} dx$ must be equal to the width of the target cell $m(x')dx$.

\section{Single degree of freedom}

\begin{defn}\label{sdof}
A degree of freedom is a label set $X$ given by a pair of homogeneous and orthogonal state variables $P$ and $Q$. We call these conjugate variables. A state variable is said conjugate of the first iff together they form a pair of conjugate variables.
\end{defn}

\begin{prop}\label{sdofMap}
Let $f: X \leftrightarrow X$ be a bijective map on a degree of freedom. Then $dq' \wedge dp' = dq \wedge dp$.
\end{prop}

This is the equivalent of \ref{widthMapping} for two state variable. The density $\rho(p,q)$ will be defined on cells of infinitesimal area proportional to $dq \wedge dp$. When mapping one cell to another, the infinitesimal area will remain the same. The area in phase space of one degree of freedom can then be used as a measure for the cardinality of the labels. \ref{labelsCount} becomes area conservation for conjugate variables. Area conservation is equivalent to requiring the invariance of the vector product.

\begin{prop}\label{sdofInvariance}
Let $v$ and $w$ be two vectors defined on the tangent space of the manifold identified by two conjugate variables. Let
\begin{align*}
\omega_{\alpha, \beta} = \left[
  \begin{array}{cc}
    0 & 1 \\
    -1 & 0 \\
  \end{array}
\right] \\
\end{align*}
then $v'^{\alpha} \omega_{\alpha, \beta} w'^{\beta}=v^{\alpha} \omega_{\alpha, \beta} w^{\beta}$ under a bijective map.
\end{prop}

\begin{lem}\label{genAntisim}
Let $v$ and $w$ be two vectors. Let $v^{\alpha} \omega_{\alpha, \beta} w^{\alpha}$ be an antisymmetric product conserved under a continuous transformation parameterized by $t$. We can then define a function $H$ such that given $S^{\alpha} \equiv d_{t}x^{\alpha}$, the vector field that tells us how the state variables change, $S_{\beta} \equiv S^{\alpha} \omega_{\alpha, \beta}$, we have $S_{\alpha} = \partial_{\alpha}H$.
\end{lem}

Simply applying the vector transformation rules under continuous transformation we have:
\begin{align*}
v^{\alpha} \omega_{\alpha, \beta} w^{\beta} &= v'^{\alpha} \omega_{\alpha, \beta} w'^{\beta}  \\
&= (v^{\alpha} + \partial_{\gamma} S^{\alpha} dt v^{\gamma}) \omega_{\alpha, \beta} ( w^{\beta} + \partial_{\delta} S^{\beta} w^{\delta} dt) \\
&= v^{\alpha} \omega_{\alpha, \beta} w^{\beta} + (\partial_{\gamma} S^{\alpha} v^{\gamma} \omega_{\alpha, \beta} w^{\beta} \\
 &+ v^{\alpha} \omega_{\alpha, \beta} \partial_{\delta} S^{\beta} w^{\delta}) dt + O(dt^2)
\end{align*}
\begin{align*}
v^{\gamma} w^{\beta} \partial_{\gamma} S_{\beta} - v^{\alpha} w^{\delta} \partial_{\delta} S_{\alpha} = 0
\end{align*}
\begin{align*}
\partial_{\alpha} S_{\beta} - \partial_{\beta} S_{\alpha} &= curl(S_{\alpha}) = 0 \\
S_{\alpha} &= \partial_{\alpha}H
\end{align*}

\begin{prop}\label{sdofHam}
The evolution for a single degree of freedom is given by:
\begin{align*}
d_{t}q &= \partial_{p} H \\
d_{t}p &= - \partial_{q} H
\end{align*}
\end{prop}

Simply expand \ref{genAntisim} with the metric defined in \ref{sdofInvariance}. We recognise Hamilton's equations for one degree of freedom.

\section{Multiple degrees of freedom}

\begin{defn}\label{mdof}
Two degrees of freedom are said independent if the corresponding label set $X^1$ and $X^2$ are independent.
\end{defn}

\begin{prop}\label{mdofInvariance}
Let $v$ and $w$ be two vectors defined on the tangent space of the manifold identified by two independent degrees of freedom. Let $\alpha$ and $\beta$ be indexes for the state variables $q^1, p^1, q^2, p^2$. Let
\begin{align*}
\omega_{\alpha, \beta} =  \left[
  \begin{array}{cc}
    1 & 0 \\
    0 & 1 \\
  \end{array}
\right] \otimes \left[
  \begin{array}{cc}
    0 & 1 \\
    -1 & 0 \\
  \end{array}
\right] =
\left[
  \begin{array}{cccc}
    0 & 1 & 0 & 0 \\
    -1 & 0 & 0 & 0 \\
    0 & 0 & 0 & 1 \\
    0 & 0 & -1 & 0 \\
  \end{array}
\right] \\
\end{align*}
then $v'^{\alpha} \omega_{\alpha, \beta} w'^{\beta}=v^{\alpha} \omega_{\alpha, \beta} w^{\beta}$ under a bijective map.
\end{prop}

The orthogonality of the labels corresponds to orthogonality in phase space, which the mapping needs to preserve across independent degrees of freedom. The cardinality of labels within each degree of freedom corresponds to the area in phase space\footnote{We assume we are using the same unit across d.o.f.}, which the mapping also needs to preserve. This is equivalent to require the conservation of the scalar product across independent degrees of freedoms, while still requiring conservation of the vector product within. That leads us to the metric defined by \ref{mdofInvariance}.
The metric generalizes \ref{sdofInvariance} to give us the cardinality of labels defined on the area given by two arbitrary vectors. For an infinitesimal region, this corresponds to $dq^1 \wedge dp^1 + dq^2 \wedge dp^2$, the sum of the projections on the independent planes.

\begin{prop}\label{mdofHam}
The evolution for a multiple degrees of freedom is given by:
\begin{align*}
d_{t}q^i &= \partial_{p^i} H \\
d_{t}p^i &= - \partial_{q^i} H
\end{align*}
\end{prop}

Expand \ref{genAntisim} with the metric defined in \ref{mdofInvariance}. We recognise Hamilton equations for multiple degrees of freedom.

\section{Time dependence}
So far we have assumed that both state labeling and mapping do not change in time. In that case, we need to to use time as a label. We introduce an extra degree of freedom.

\begin{defn}\label{tdof}
The temporal degree of freedom is a label set $X$ given by the pair of conjugate variables $T$ and $E$. We call extended phase space the outer product between phase space and the temporal degree of freedom.
\end{defn}

\begin{prop}\label{tdofMonotonic}
Let $s$ be the parameter of a trajectory in the extended phase space of a deterministic and reversible system. The trajectory must be continuous. There must exist a strictly monotonic function $t(s)$. States are connected by a trajectory where either $dt/ds>0$ (particle states) or $dt/ds<0$ (anti-particle states).
\end{prop}

The trajectory has to be continuous in both spatial and temporal variables because of \ref{continuousMapping}. Since determinism and reversibility are defined in time, the trajectory must traverse all times once and only once: we must have an invertible mapping between $t$ and $s$, which means we must have a strictly monotonic $t(s)$. $dt/ds$ along a trajectory cannot change sign, so we have the division between particle and anti-particle states. Note that since the parametrization is conventional and can be changed to $s'=-s$, what we call particle and anti-particle states is also conventional. What is physical and not conventional, though, is that particle and anti-particle states cannot be connected by deterministic and reversible evolution.

\begin{prop}\label{tdofInvariant}
Let $v$ and $w$ be two vectors defined on the tangent space of the extended phase space for one degree of freedom. Let $\alpha$ and $\beta$ be indexes for the state variables $t, e, q, p$. Let
\begin{align*}
\omega_{\alpha, \beta} =  \left[
  \begin{array}{cc}
    -1 & 0 \\
    0 & 1 \\
  \end{array}
\right] \otimes \left[
  \begin{array}{cc}
    0 & 1 \\
    -1 & 0 \\
  \end{array}
\right]
= \left[
  \begin{array}{cccc}
    0 & -1 & 0 & 0 \\
    1 & 0 & 0 & 0 \\
    0 & 0 & 0 & 1 \\
    0 & 0 & -1 & 0 \\
  \end{array}
\right] \\
\end{align*}
then $v'^{\alpha} \omega_{\alpha, \beta} w'^{\beta}=v^{\alpha} \omega_{\alpha, \beta} w^{\beta}$ under detrev transformation.
\end{prop}

$\langle T, E \rangle$ are not independent labels from $\langle Q, P \rangle$ as they do not define new states. So they are not necessarily orthogonal in the extended phase space. Looking back at \ref{disclabelsoverr}, cells need to be defined on the plane where $\langle Q, P \rangle$ (maximally) changes: this is not the plane of constant $\langle E, T \rangle$ (they are not orthogonal) where $dq \wedge dp$ is defined, but the plane perpendicular to constant $\langle Q, P \rangle$ where $dt \wedge de$ is defined. On that plane we can properly count states and define our invariant.

As in the case of independent d.o.f., we have three planes of conjugate variables forming a right triangle-like relationship with one side invariant.
\begin{align*}
&dq^1 \wedge dp^1 + dq^2 \wedge dp^2 = k \\
&dt \wedge de + k = dq \wedge dp
\end{align*}
In the previous case, the right angle was between the two independent degree of freedom. In this case, the right angle is between the invariant and the plane of constant $\langle Q, P \rangle$ where $dt \wedge de$ is defined. We rewrite it as $dq \wedge dp - dt \wedge de = k$. Which corresponds to the Minkowski product across d.o.f. and the vector product within. The metric, with a space-like convention, still gives us the cardinality of labels within a degree of freedom.

\begin{prop}\label{tdofHam}
The evolution for time varying multiple degrees of freedom is given by:
\begin{align*}
d_{s}t &= - \partial_{e} \mathcal{H} \\
d_{s}e &= \partial_{t} \mathcal{H} \\
d_{s}q^i &= \partial_{p^i} \mathcal{H} \\
d_{s}p^i &= - \partial_{q^i} \mathcal{H}
\end{align*}
\end{prop}

Expand \ref{genAntisim} with the metric defined in \ref{tdofInvariant}, with the parameter $s$ instead of $t$ and generator $\mathcal{H}$ instead of $H$. We recognise Hamilton equations in the extended phase space.

It should not be a surprise that the equations do not mention $c$. In fact, nothing says that all $q^i$ represent space, and with the same unit, and that the laws of motion are invariant in all inertial frames. The only requirement we have is that the areas of each degrees of freedom represent the same cardinality for labels.

\begin{prop}\label{tdofConstrain}
The evolution is constrained by $\mathcal{H}=k$, the generating function defined in \ref{genAntisim} generalized for the extended phase space.
\end{prop}

Given N degrees of freedom, phase space is $\mathbb{R}^{2*N}$. Extended phase space is $\mathbb{R}^{2*N + 2}$. Phase space evolved in time is $\mathbb{R}^{2*N + 1}$. Since $\mathcal{H}$ is constant through the evolution, it can serve both as the generating function and as the evolution constrain. By convention, we can set $\mathcal{H}=0$ without loss of generality as changing $\mathcal{H}$ by a constant does not change the equation of motion.

\textbf{Example}. We wrap up with an example. Let $\mathcal{H}= mc^2 + ((p^i)^2 - E^2/c^2) / 2m$. The constant term represents the rest energy of a free particle, the conjugate of which is proper time. The parametrization of a particle state is $s=\tau$ while the parametrization for an anti-particle state is $s=-\tau$. If we apply \ref{tdofHam} we have:
\begin{align*}
d_{s}t &= E / mc^2 \\
d_{s}E &= 0 \\
d_{s}q^i &= p^i / m \\
d_{s}p^i &= 0
\end{align*}
For particles, we have:
\begin{align*}
E / c &= m c d_{\tau}t = m U^0 = P^0 \\
p^i &= m d_{\tau}q^i = m U^i = P^i \\
\end{align*}
so we recognize the four-momentum $P^\alpha = {E/c, p^i}$. For anti-particles:
\begin{align*}
- E / c &= m c d_{\tau}t = m U^0 = P^0 \\
- p^i &= m d_{\tau}q^i = m U^i = P^i \\
\end{align*}
we end up with a minus sign between the four-momentum and the conjugate variables $P^\alpha = {-E/c, -p^i}$.

\section{Conclusion}
By deriving Hamiltonian mechanics from simple definitions of labels, states, determinism and reversibility we have given more direct physical meaning to phase space and its geometric properties, and shown that a good part of the Hamiltonian framework can stand on its own, without Lagrangians. No mathematical breakthrough is revealed here, yet I am not aware of any work that brings all the pieces of the puzzle in quite this way: so much derived from so little.

The hope is that, by continuing in this approach, we can shed more light on why nature has the laws it has; and show that they are not some arbitrary rules, but indeed necessary given few simple assumptions.

\begin{thebibliography}{5}

\bibitem{extend} No bib right now.

\end{thebibliography}

\end{document}
