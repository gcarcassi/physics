\documentclass[aps,pra,10pt,twocolumn,floatfix,nofootinbib]{revtex4-1}

\usepackage{bbm}
\usepackage{amsmath}
\usepackage{amssymb}
\usepackage{graphicx}
\usepackage{amsfonts}
\usepackage{amsthm}

\newtheorem{thm}{Theorem}[section]
\newtheorem{cor}[thm]{Corollary}
\newtheorem{lem}[thm]{Lemma}
\newtheorem{prop}[thm]{Proposition}

\theoremstyle{definition}
\newtheorem{defn}[thm]{Definition}
\newtheorem*{assump1}{Classical assumption}
\newtheorem*{assump2}{Determinism and Reversibility assumption}

\begin{document}

\section{Notes for Dirac (spin) equation}
\begin{align*}
&\xi^a\equiv\{t, q^i, -E, p_i\} \\
&\xi^a(s)\\
&\omega \equiv  dq^i \wedge dp_i - dt \wedge dE \\
&\omega_{ab} =  \left[
\begin{array}{cc}
	0 & 1 \\
	-1 & 0 \\
\end{array}
\right] \otimes \left[
\begin{array}{cccc}
	1 & 0 & 0 & 0\\
	0 & 1 & 0 & 0\\
	0 & 0 & 1 & 0 \\
	0 & 0 & 0 & 1 \\
\end{array}
\right] \\
&\mathcal{H}\equiv\mathcal{H}(\xi^a) \\
&\mathcal{H} = 0 \\
&d_s \mathcal{H} = 0 \\
&d_s\xi^a\omega_{a b} = \partial_b\mathcal{H}  \\
\end{align*}

For free particle:
\begin{align*}
\mathcal{H} &= \frac{1}{2m} \mu^{\alpha\beta} p_\alpha p_\beta + \frac{1}{2} mc^2
\end{align*}

To quantum (Klein Gordon):
\begin{align*}
p_\alpha &= \imath \hbar \partial_\alpha \\
\mathcal{H} &= \frac{\hbar^2}{2m} (\mu^{\alpha\beta} \partial_\alpha \partial_\beta + \frac{m^2c^2}{\hbar^2} )
\end{align*}

For free particle:
\begin{align*}
\mathcal{H} &= \frac{1}{2m} \mu^{\alpha\beta} p_\alpha p_\beta + \frac{1}{2} mc^2
\end{align*}

For spin 1/2 (Dirac):
\begin{align*}
u^\alpha s_\beta &= 0 \\
u^\alpha &= c \frac{1}{2} (n_R + n_L) \\
s^\alpha &= \frac{\hbar}{2} \frac{1}{2} (n_R - n_L) \\
\mathcal{H} &= u^\alpha p_\alpha + mc^2 \\
u^\alpha &= c \gamma^\alpha \\
p_\alpha &= \imath \hbar \partial_\alpha \\
\mathcal{H} &= c (\gamma^\alpha \imath \hbar \partial_\alpha + mc)
\end{align*}

For non relativistic spin 1/2:
\begin{align*}
\eta^a &\equiv \{\theta^{xy}, s_z\} \\
\omega &= d\theta_{xy} \wedge ds_z \\
s_i &= ( \sqrt{1-s_z^2} cos \theta^{xy}, \sqrt{1-s_z^2} sin \theta^{xy}, s_z) \\
\{ s_x, s_y \} &= \frac{\partial s_x}{\partial \theta^{xy}} \frac{\partial s_y}{\partial s_z} - \frac{\partial s_y}{\partial \theta^{xy}} \frac{\partial s_x}{\partial s_z} \\
&= \sqrt{1-s_z^2} (-1) sin \theta^{xy} \frac{-s_z}{\sqrt{1-s_z^2}}sin \theta^{xy}  \\
&- \sqrt{1-s_z^2} cos \theta^{xy} \frac{-s_z}{\sqrt{1-s_z^2}}cos \theta^{xy} \\
&= s_z \\
\{ s_y, s_z \} &= \frac{\partial s_y}{\partial \theta^{xy}} \frac{\partial s_z}{\partial s_z} - \frac{\partial s_z}{\partial \theta^{xy}} \frac{\partial s_y}{\partial s_z}\\
&= \sqrt{1-s_z^2} cos \theta^{xy} = s_x\\
\{ s_z, s_x \} &= \frac{\partial s_z}{\partial \theta^{xy}} \frac{\partial s_x}{\partial s_z} - \frac{\partial s_x}{\partial \theta^{xy}} \frac{\partial s_z}{\partial s_z}\\
&= - \sqrt{1-s_z^2} (-1) sin \theta^{xy} = s_y \\
\{ s_i, s_j \} &= \epsilon_{ijk} s_k \\
H &= B^i s_i \\
\{ s_i, H \} &= \epsilon_{ijk}B^i s_k \\
\end{align*}


\section{Notes for classical Field Theory}

Use roman letters for phase space indexes. Use greek letters for space-time indexes.

Particle mechanics:
\begin{align*}
&\xi^a\equiv\{t, q^i, E, p_i\} \\
&\xi^a(s)\\
&\omega \equiv  dq^i \wedge dp_i - dt \wedge dE \\
&\omega_{ab} =  \left[
  \begin{array}{cc}
    0 & 1 \\
    -1 & 0 \\
  \end{array}
\right] \otimes \left[
  \begin{array}{cccc}
    -1 & 0 & 0 & 0\\
    0 & 1 & 0 & 0\\
    0 & 0 & 1 & 0 \\
    0 & 0 & 0 & 1 \\
  \end{array}
\right] \\
&\mathcal{H}\equiv\mathcal{H}(\xi^a) \\
&\mathcal{H} = 0 \\
&d_s \mathcal{H} = 0 \\
&S^a=d_s\xi^a \\
&S_a = \partial_a\mathcal{H} \\
&S_b = S^a \omega_{a, b} \\
&d_{s}t = - \partial_{E} \mathcal{H} \\
&d_{s}E = \partial_{t} \mathcal{H} \\
&d_{s}q^i = \partial_{p_i} \mathcal{H} \\
&d_{s}p_i = - \partial_{q^i} \mathcal{H} \\
\end{align*}

Textbook time-based field theory.
\begin{align*}
&x^\alpha\equiv\{ct, x^i\} \\
&\phi\equiv\phi(x^\alpha) \\
&\pi\equiv\pi(x^\alpha) \\
&\mathcal{H}\equiv\mathcal{H}(\pi, \phi, \partial_i\phi, x^a) \\
&H\equiv \int \mathcal{H} d^3x \\
&\partial_{t}\phi = \partial_{\pi} \mathcal{H} \\
&\partial_{t}\pi = - \partial_{\phi} \mathcal{H} + \partial_i \partial_{\partial_i\phi} \mathcal{H} \\
&\partial_{t}\phi = \delta_{\pi} H \\
&\partial_{t}\pi = - \delta_{\phi} H \\
\end{align*}

Covariant field theory.
\begin{align*}
&x^\alpha\equiv\{ct, x^i\} \\
&\phi\equiv\phi(x^\alpha) \\
&\pi^\alpha\equiv\pi^\alpha(x^\beta) \\
&\mathcal{H}\equiv\mathcal{H}(\pi^\alpha, \phi, x^a) \\
&\partial_{\alpha}\phi = \partial_{\pi^\alpha} \mathcal{H} \\
&\partial_{\alpha}\pi^\alpha = - \partial_{\phi} \mathcal{H} \\
\end{align*}

What I need ($s$ is parameter of the evolution, $s_\perp^i$ are defined on surfaces of equal $s$, and therefore are locally perpendicular):
\begin{align*}
&x^\alpha\equiv\{ct, x^i\} \\
&s\equiv s(x^a) \\
&\partial_ts \neq 0 \\
&\partial_\alpha s \partial^\alpha s = -1 \\
&\phi\equiv\phi(s, s_\perp^i) \\
&\pi\equiv\pi(s, s_\perp^i) \\
&\omega \equiv \int \delta\phi \wedge \delta\pi ds_\perp^i \\
&\mathcal{H}\equiv\mathcal{H}(\pi, \phi, s_{\perp}^i) \\
&\partial_s\mathcal{H}=0 \\
&\partial_s\phi = \partial_{\pi} \mathcal{H} \\
&\partial_s\pi = - \partial_{\phi} \mathcal{H} \\
\end{align*}

Possible link to covariant field theory:
\begin{align*}
&\pi\partial_s\phi = \pi\partial_s x^\alpha \partial_\alpha \phi \\
&\pi^\alpha = \pi\partial_s x^\alpha \\
&\mathcal{H} = \pi\partial_s\phi - \mathcal{L} \\
&= \pi^\alpha \partial_\alpha \phi - \mathcal{L} \\
&\omega = \int \delta\phi \wedge \delta\pi \partial_s x^\alpha n_\alpha d\Sigma \\
&= \int \delta\phi \wedge \delta\pi^\alpha n_\alpha d\Sigma \\
\end{align*}

\begin{align*}
&\partial_\alpha\pi^\alpha = \partial_\alpha \pi \partial_s x^\alpha + \pi\partial_\alpha \partial_s x^\alpha \\
&\partial_\alpha \partial_s x^\alpha = \partial_s \partial_\alpha x^\alpha = \partial_s \delta_\alpha^\alpha = 0 \\
&\partial_\alpha\pi^\alpha = \partial_\alpha \pi \partial_s x^\alpha = \partial_s\pi = - \partial_{\phi} \mathcal{H} \\
&\partial_\alpha\phi  = \partial_s \phi / \partial_s x^\alpha = \partial_\pi \mathcal{H} / \partial_s x^\alpha =
\partial_{\pi \partial_s x^\alpha} \mathcal{H} = \partial_{\pi^\alpha} \mathcal{H} \\
\end{align*}

Link between energy-momentum (generator of space-time translation) and Hamiltonian (generator for evolution along s):
\begin{align*}
&T_\alpha^\mu=\delta_\alpha^\mu \mathcal{H} + \pi^\mu \partial_\alpha \phi - \delta_\alpha^\mu \pi^\beta \partial_\beta \phi \\
&T_\alpha^\mu \partial_s x^\alpha = \delta_\alpha^\mu \mathcal{H} \partial_s x^\alpha + \pi \partial_s x^\mu \partial_\alpha \phi \partial_s x^\alpha - \delta_\alpha^\mu \pi \partial_s x^\beta \partial_\beta \phi \partial_s x^\alpha \\
&T_\alpha^\mu \partial_s x^\alpha = \mathcal{H} \partial_s x^\mu + \pi \partial_s x^\mu \partial_s \phi - \pi \partial_s \phi \partial_s x^\mu \\
&T_\alpha^\mu \partial_s x^\alpha ds = \mathcal{H} \partial_s x^\mu ds \\
\end{align*}


Is this true (all taken at equal $s_\perp^i$):
\begin{align*}
&1 = \partial_s s  = \partial_\alpha s \partial_s x^\alpha \\
&-1 = \partial_\alpha s \partial^\alpha s \\
&\partial^\alpha s = -\partial_s x^\alpha \\
\end{align*}

If so:
\begin{align*}
&\pi^\alpha = -\pi\partial^\alpha s \\
\end{align*}
$\pi^\alpha$ points in the direction of max $s$ increase, and it's modulus is $\pi$. It's divergence is the change of $\pi$ along $s$. Which we can use to translate the equations of motion. Note that all derivatives of $\mathcal{H}$ are taken at $x^\alpha$ constant (we are changing the value of the field, not the position).




\end{document}
